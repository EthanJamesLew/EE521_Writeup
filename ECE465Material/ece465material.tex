\chapter{ECE 465 Material}

\section{Homeworks}

\subsection{Homework 2}

\subsubsection{Problem}

In a DSP application, it requires an A/D converter with a 1-volt full scale. 
It also requires that the RMS value of the quantization noise to be less than 
0.001 volt.

\begin{enumerate}
    \item What is the required resolution (how many bits) of this converter? 
    What is the actual RMS value of the noise in this converter? 
    \item  If one wants to record a 5-minute audio signal using that converter 
    with a sampling rate of 44 kHz, how many bits of disk space is required?
\end{enumerate}

\subsubsection{Solution}
Assuming a constant noise function, $E$, being the ideal quantizer. The RMS voltage can be expressed,

\begin{equation}
{E_{rms}} = \sqrt {\frac{1}{Q}\int\limits_{ - Q/2}^{Q/2} {{{\left| {E(e)} \right|}^2}de} }  = \sqrt {\frac{1}{Q}\int\limits_{ - Q/2}^{Q/2} {{e^2}de} }  = \frac{Q}{{\sqrt {12} }}
\end{equation}

A linear relationship exists between $E_{rms}$ and $Q$. Therefore, the specified noise requirement can be solved for

\begin{equation}
\begin{aligned}
  {E_{rms}} = \frac{Q}{{\sqrt {12} }} &\leqslant 0.001 \hfill \\
   \Rightarrow Q &\leqslant 0.003464 \hfill \\ 
\end{aligned} 
\end{equation}

The minimum number of bits can be solved for

\begin{equation}
\begin{aligned}
  {B_{\min }} &= \left\lceil {\lg \left( {\frac{R}{Q}} \right)} \right\rceil  \hfill \\
   &= \left\lceil {\lg \left( {\frac{1}{{0.0035}}} \right)} \right\rceil  \hfill \\
   &= 9 \hfill \\ 
\end{aligned} 
\end{equation}

The ADC can be improved by using a more typical value of 10 bits.

Using 9 bits, the new RMS is

\begin{equation}
\begin{aligned}
  E_{rms} &= \frac{R}{{{2^B}\sqrt{12}}} \hfill \\
   &= \frac{1}{{{2^9}\sqrt {12} }} \hfill \\
   &\approx 0.00056 \hfill \\ 
\end{aligned} 
\end{equation}


For the disk space per audio recording, the number of storage bits per signal is
\begin{equation}
\begin{aligned}
  N\left( B \right) &= \left( {\frac{{44k{\text{ samples}}}}{{\text{s}}}} \right)\left( {\frac{{60{\text{ s}}}}{{{\text{min}}}}} \right)\left( {\frac{{5{\text{ min}}}}{{{\text{signal}}}}} \right)\left( {\frac{{B{\text{ bits}}}}{{{\text{sample}}}}} \right) \hfill \\
   &= 13.2{\text{M}} \cdot B \hfill \\ 
\end{aligned} 
\end{equation}

Accordingly,

\begin{equation}
N(9)=118.8 \text{ Mb}= 118.8 \text{ Mb}
\end{equation}

This would consume 29k blocks on a hard disc with a minimum addressable unit of 4096 bits.

\subsection{Homework 3}

\subsubsection{Problem}

Determine the z-transform of the following sequences:

\begin{equation}
x[n] = {\left( {0.7} \right)^n}u[n]
\end{equation}
    
\begin{equation}
x[n] = 4\delta [n] + {\left( {0.5} \right)^n}u[n - 3]
\end{equation}

\subsubsection{Solution}

The z-transform is defined as follow
\begin{equation}
X(z) = \sum\limits_{n =  - \infty }^\infty  {x[n]{z^{ - n}}} 
\end{equation}

Substitute and reduce for the given $x[n]$

\begin{equation}
\begin{aligned}
X(z) &= \sum\limits_{n =  - \infty }^\infty  {x[n]{z^{ - n}}}  \hfill \\
&= \sum\limits_{n =  - \infty }^\infty  {{{\left( {0.7} \right)}^n}u[n]{z^{ - n}}}  \hfill \\
&= \sum\limits_{n = 0}^\infty  {{{\left( {0.7} \right)}^n}{z^{ - n}}}  \hfill \\
&= {\sum\limits_{n = 0}^\infty  {\left( {\frac{{0.7}}{z}} \right)} ^{n}} \hfill \\ 
\end{aligned} 
\end{equation}

This reduces into a geometric series. Therefore the solution is
\begin{equation}
X(z) = \frac{z}{{z - 0.7}},\left| z \right| > 0.7
\end{equation}

Using similar methods,
	\begin{equation}
	\begin{aligned}
  X(z) &= \sum\limits_{n =  - \infty }^\infty  {x[n]{z^{ - n}}}  \hfill \\
   &= \sum\limits_{n =  - \infty }^\infty  {\left( {4\delta [n] + {{\left( {0.5} \right)}^n}u[n - 3]} \right){z^{ - n}}}  \hfill \\
   &= 4\sum\limits_{n =  - \infty }^\infty  {\delta [n]} {z^{ - n}} + \sum\limits_{n = 3}^\infty  {{{\left( {0.5} \right)}^n}{z^{ - n}}}  \hfill \\
   &= 4 - \sum\limits_{n = 0}^2 {{{\left( {\frac{{0.5}}{z}} \right)}^n}}  + \sum\limits_{n = 0}^\infty  {{{\left( {\frac{{0.5}}{z}} \right)}^n}}  \hfill \\
   &= 4 - 1 - \frac{{0.5}}{z} - \frac{{0.25}}{{{z^2}}} + \sum\limits_{n = 0}^\infty  {{{\left( {\frac{{0.5}}{z}} \right)}^n}}  \hfill \\
\end{aligned} 
	\end{equation}
	Therefore, the solution is
	\begin{equation}
	X(z)= 3 - \frac{{0.25}}{z}\left( {2 - \frac{1}{z}} \right) + \frac{z}{{z - 0.5}},\left| z \right| > 0.5 
    \end{equation}

\subsubsection{Problem}

An LTI system has the following impulse response 

\begin{equation}
h[n]=\cos\left( n\frac{\pi}{5}\right)u[n].
\end{equation}

Is the system stable? Why? Repeat the analysis for

\begin{equation}
    h[n]=0.9^nu[n].
\end{equation}


\subsubsection{Solution}
    
The absolute sum of $h[n]$ is

\begin{equation}
\begin{aligned}
\sum\limits_{n =  - \infty }^\infty  {\left| 0.9^n u[n] \right|}  &= \sum\limits_{n = 0}^\infty  {0.9^n } \hfill \\
   &= \frac{1}{{0.1}} \hfill \\
   &= 10 \hfill \\ 
\end{aligned}
\end{equation}

The system described by $h[n]$ is stable if and only if it is absolutely summable. That is,
\begin{equation}
\sum\limits_{n =  - \infty }^\infty  {\left| {h[n]} \right| < \infty } 
\end{equation}

The system is causal, so

\begin{equation}
\sum\limits_{n =  - \infty }^\infty  {\left| {\cos \left( {\frac{{n\pi }}{5}} \right)u[n]} \right|}  = \sum\limits_{n = 0}^\infty  {\left| {\cos \left( {\frac{{n\pi }}{5}} \right)} \right|} 
\end{equation}

The terms in the sum repeat periodically, allowing the sum to be decomposed

\begin{equation}
\begin{aligned}
\sum\limits_{n =  - \infty }^\infty  {\left| {\cos \left( {\frac{{n\pi }}{5}} \right)u[n]} \right|}  =& \sum\limits_{n = 0}^\infty  {\left| {\cos \left( 0 \right)} \right|}  + \sum\limits_{n = 0}^\infty  {\left| {\cos \left( {\frac{\pi }{5}} \right)} \right|}  + \sum\limits_{n = 0}^\infty  {\left| {\cos \left( {\frac{{2\pi }}{5}} \right)} \right|}  + \hfill \\
&\sum\limits_{n = 0}^\infty  {\left| {\cos \left( {\frac{{3\pi }}{5}} \right)} \right|}  + \sum\limits_{n = 0}^\infty  {\left| {\cos \left( {\frac{{4\pi }}{5}} \right)} \right|} \hfill \\
\end{aligned}
\end{equation}

One of the sums can be expressed as
\begin{equation}
\sum\limits_{n = 0}^\infty  {\left| {\cos \left( 0 \right)} \right|}  = \sum\limits_{n = 0}^\infty  1 
\end{equation}

which diverges. Therefore, the original sum must also diverge. \textbf{The system is unstable}.
\\ 
The sum converges. \textbf{The system is stable}.

\section{Exams}

\subsection{Midterm}

\subsubsection{Problem}

Consider the following signal

\begin{equation}
    x(t) = \sin \left(10000\pi t \right) + \sin \left(20000\pi t \right) + \sin \left(60000\pi t \right)
\end{equation}

The signal is pre-filtered by an anti-aliasing filter $H(f)$, 
and then sampled at the sampling rate of 40,000 Hz. 
The samples are then reconstructed by an ideal reconstruction filter to 
reproduce the analog signal. Find the output signal for the following cases:

\begin{enumerate}
    \item $H(f) = 1$ i.e. no prefiltering
    \item $H(f)$ is an ideal LPF with $f_c = 20,000$ Hz.
    \item $H(f)$ is a filter with flat response up to 20,000 Hz, and attenuate at the rate of 60dB per Octave beyond 
    20,000 Hz.
\end{enumerate}


\subsubsection{Solution}

Constructing the graph in figure \ref{fig:465midalias}, it is evident that 

\begin{equation}
    y(t) = \sin \left(10000\pi t \right).
\end{equation}

Referring to figure \ref{fig:465midlpf}, for the ideal LPF,

\begin{equation}
    y(t) = \sin \left(10000\pi t \right) + \sin \left(20000\pi t \right).
\end{equation}

For the case of the non-ideal LPF, calculate how many octaves away the 
30 kHz tone is,

\begin{equation}
    \lg \left( {\frac{{30}}{{20}}} \right) \approx 0.585{\text{ octaves}}
\end{equation}

The attenuation, then, is

\begin{equation}
    60\lg \left( {\frac{{30}}{{20}}} \right) \approx 35.1{\text{ dB }} \approx {\rm{ 0}}{\rm{.0176}}
\end{equation}

Accordingly, for the non-ideal LPF,

\begin{equation}
    y(t) = \sin \left(10000\pi t \right) + 0.982 \sin \left(20000\pi t \right).
\end{equation}


\begin{figure}
    \centering
\begin{tikzpicture}
    \makeatletter
    \begin{axis}[axis lines=middle,xmin=-60000,xmax=60000,ymin=-2,ymax=2, xlabel={$f$},
        ylabel={$Y\left(f \right)$}]
    \addplot +[dirac] coordinates {(-5000,1) (5000,-1) (-10000,1) (10000,-1) (-30000,1) (30000,-1)};
    \addplot +[dirac] coordinates {(-5000+40000,1) (5000+40000,-1) (-10000+40000,1) (10000+40000,-1) (-30000+40000,1) (30000+40000,-1)};
    \addplot +[dirac] coordinates {(-5000-40000,1) (5000-40000,-1) (-10000-40000,1) (10000-40000,-1) (-30000-40000,1) (30000-40000,-1)};
    \addplot +[thick, mark=none, dashed,  const plot] 
    coordinates {(-60000, 0) (-20000,0) (-20000, 1) (20000, 1) (20000, 0) (60000, 0)};
    \end{axis}
    \end{tikzpicture}
\caption{Spectrum of $y(t)$ produced from no prefiltering. Note that the aliases cancel out frequencies present in the Nyquist window.}
\label{fig:465midalias}
\end{figure}

\begin{figure}
    \centering
\begin{tikzpicture}
    \makeatletter
    \begin{axis}[axis lines=middle,xmin=-60000,xmax=60000,ymin=-2,ymax=2, xlabel={$f$},
        ylabel={$Y\left(f \right)$}]
    \addplot +[dirac] coordinates {(-5000,1) (5000,-1) (-10000,1) (10000,-1) };
    \addplot +[dirac] coordinates {(-5000+40000,1) (5000+40000,-1) (-10000+40000,1) (10000+40000,-1) };
    \addplot +[dirac] coordinates {(-5000-40000,1) (5000-40000,-1) (-10000-40000,1) (10000-40000,-1) };
    \addplot +[thick, mark=none, dashed,  const plot] 
    coordinates {(-60000, 0) (-20000,0) (-20000, 1) (20000, 1) (20000, 0) (60000, 0)};
    \end{axis}
    \end{tikzpicture}
\caption{Spectrum of $y(t)$ produced from an ideal LPF with $f_c$ = 20,000 Hz. Note that the 10,000 Hz tone is preserved.}
\label{fig:465midlpf}
\end{figure}

\subsubsection{Problem}
An LTI system has the following impulse response
\begin{enumerate}
    \item $h[n] = (-1)^n u[n]$. Is this system stable? Why?
    \item Repeat with $h[n] = (0.9)^n \cos \left( \frac{n \pi}{5} \right) u[n]$
\end{enumerate}
\subsubsection{Solution}

A system is said to be stable if its impulse response is absolutely stable.

\begin{equation}
    \sum\limits_{n =  - \infty }^\infty  {\left| {h[n]} \right|}  < \infty 
\end{equation}

For 1), 

\begin{equation}
    \begin{aligned}
        \sum\limits_{n =  - \infty }^\infty  {\left| {{{( - 1)}^n}u[n]} \right|}  &= \sum\limits_{n = 0}^\infty  {\left| {{{( - 1)}^n}} \right|} \\
         &= \infty 
        \end{aligned}
\end{equation}

For 2),

\begin{equation}
    \begin{aligned}
            \sum\limits_{n =  - \infty }^\infty  {\left| {{{(0.9)}^n}\cos \left( {\frac{{n\pi }}{5}} \right)u[n]} \right|}  &< \sum\limits_{n = 0}^\infty  {\left| {{{(0.9)}^n}} \right|} \\
             &= \frac{1}{{1 - 0.9}}
        \end{aligned}
\end{equation}

Being less than 10, the system is stable.

%\begin{equation}
%    \begin{aligned}
%            \sum\limits_{n =  - \infty }^\infty  {\left| {{{(0.9)}^n}\cos \left( {\frac{{n\pi }}{5}} \right)u[n]} \right|}  =& \sum\limits_{n = 0}^\infty  {\left| {{{(0.9)}^n}\cos \left( {\frac{{n\pi }}{5}} \right)} \right|} \\
%             =& 2\sum\limits_{n = 0}^\infty  {\left| {{{(0.9)}^{5n}}} \right|}  + \cos \left( {\frac{\pi }{5}} \right)\sum\limits_{n = 0}^\infty  {\left| {{{(0.9)}^{5n + 1}} + {{(0.9)}^{5n + 6}}} \right|}  + \cos \left( {\frac{{2\pi }}{5}} \right)\sum\limits_{n = 0}^\infty  {\left| {{{(0.9)}^{5n + 2}} + {{(0.9)}^{5n + 7}}} \right|} \\
%             &+ \cos \left( {\frac{{3\pi }}{5}} \right)\sum\limits_{n = 0}^\infty  {\left| {{{(0.9)}^{5n + 3}} + {{(0.9)}^{5n + 8}}} \right|}  + \cos \left( {\frac{{4\pi }}{5}} \right)\sum\limits_{n = 0}^\infty  {\left| {{{(0.9)}^{5n + 4}} + {{(0.9)}^{5n + 9}}} \right|} 
%        \end{aligned}
%\end{equation}

\subsubsection{Problem}
An LTI system has a transfer function: 

\begin{equation}
    H(z) = \frac{3 + 0.1 z^{-1}}{1-0.5z^{-1} -0.66 z^{-2}}
\end{equation}

\begin{enumerate}
    \item Find the causal impulse response $h[n]$.
    \item Find the stable impulse response $h[n]$.
\end{enumerate}

\subsubsection{Solution}

Expand $H(z)$ to facilitate solving for the inverse Z-transform, 
\begin{equation}
    H(z) = \frac{{3 + 0.1{z^{ - 1}}}}{{\left( {1 + 0.6{z^{ - 1}}} \right)\left( {1 - 1.1{z^{ - 1}}} \right)}} = \frac{2}{{1 + 0.6{z^{ - 1}}}} - \frac{{3.8}}{{1 - 1.1{z^{ - 1}}}}.
\end{equation}

So, for the causal impulse response,
\begin{equation}
    h[n] = {Z^{ - 1}}\left\{ {\frac{2}{{1 + 0.6{z^{ - 1}}}} - \frac{{3.8}}{{1 - 1.1{z^{ - 1}}}}} \right\} = \left( {2{{\left( {0.6} \right)}^n} - 3.8{{\left( {1.1} \right)}^n}} \right)u[n].
\end{equation}

For the stable impulse response,
\begin{equation}
    h[n] = {Z^{ - 1}}\left\{ {\frac{2}{{1 + 0.6{z^{ - 1}}}} - \frac{{3.8}}{{1 - 1.1{z^{ - 1}}}}} \right\} = \left( {2{{\left( {0.6} \right)}^n}u[n] + 3.8{{\left( {1.1} \right)}^n}u[ - n - 1]} \right).
\end{equation}







