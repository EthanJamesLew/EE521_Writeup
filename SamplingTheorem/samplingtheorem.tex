\chapter{The Sampling Theorem}

\section{Deriving the Sampling Theorem }



\begin{thm}{\textbf{Sampling Theorem}}
The discrete-time Fourier transform of the sampled sequence $x[n]$ relates to the continuous-time Fourier transform of the signal $x_a(t)$ as

\begin{equation}
X\left( {\Omega T} \right) = \frac{1}{T}\sum\limits_{n =  - \infty }^\infty  {{X_a}\left( {\frac{\omega }{T} + \frac{{n2\pi }}{T}} \right)}.
\end{equation}
\end{thm}

\begin{proof}
By the sampling relation,  

\begin{equation}
X\left( \omega  \right) = \sum\limits_{n =  - \infty }^\infty  {{x_a}(nT){e^{ - j\omega n}}} .
\end{equation}

This can be expressed in term of the continuous-time Fourier transformation of $x_a(t)$,

\begin{equation}
\begin{aligned}
X\left( \omega  \right) &= \sum\limits_{n =  - \infty }^\infty  {\frac{1}{{2\pi }}\int\limits_{ - \infty }^\infty  {{X_a}(\Omega ){e^{j\Omega nt}}d\Omega } {e^{ - j\omega n}}} \\
 &= \frac{1}{{2\pi }}\int\limits_{ - \infty }^\infty  {\sum\limits_{n =  - \infty }^\infty  {{X_a}\left( \Omega  \right)} {e^{ - j\omega n}}{e^{j\Omega nt}}d\Omega } .
\end{aligned}
\end{equation}

\begin{lemma} \label{comb}
Given $\omega \in [-\pi, \pi]$, 
\begin{equation}
\mathop {\lim }\limits_{N \to \infty } \sum\limits_{n =  - N}^N {{e^{jn\omega }}}  = 2\pi \delta \left( \omega  \right).
\end{equation}
\end{lemma}

\begin{proof}
Recall the geometric identity,

\begin{equation}
\begin{aligned}
\mathop {\lim }\limits_{N \to \infty } \sum\limits_{n =  - N}^N {{e^{jn\omega }}}  &= \mathop {\lim }\limits_{N \to \infty } {e^{ - j\omega N}}\sum\limits_{n = 0}^N {{e^{jn\omega }}} \\
 &= \mathop {\lim }\limits_{N \to \infty } {e^{ - j\omega N}}\frac{{1 - {e^{j\left( {2N + 1} \right)\omega }}}}{{1 - {e^{j\omega }}}}
\end{aligned}
\end{equation}

As $\text{sinc}(x) = e^{jx} - e^{-j}$,

\begin{equation}
\begin{aligned}
\mathop {\lim }\limits_{N \to \infty } \sum\limits_{n =  - N}^N {{e^{jn\omega }}}  &= \mathop {\lim }\limits_{N \to \infty } \left( {2N + 1} \right)\frac{{\text{sinc} \left( {\left( {\frac{{2N + 1}}{2}} \right)\omega } \right)}}{{\text{sinc} \left( {\frac{\omega }{2}} \right)}} \\
&= \alpha \delta(\omega)
\end{aligned}
\end{equation}

It is clear that the expression is proportional to the dirac delta function $\delta(\omega)$. Integrating the original expression over all values of $\omega$ will solve for $\alpha$:

\begin{equation}
\int\limits_{ - \pi }^\pi  {\mathop {\lim }\limits_{N \to \infty } \sum\limits_{n =  - N}^N {{e^{jn\omega }}} } d\omega  = 2\pi 
\end{equation}

Thus,

\begin{equation}
\mathop {\lim }\limits_{N \to \infty } \sum\limits_{n =  - N}^N {{e^{jn\omega }}} = 2 \pi \delta(\omega). 
\end{equation}
\end{proof}

Accordingly, from Lemma \ref{comb},

\begin{equation}
X\left( \omega  \right) = \int\limits_{ - \infty }^\infty  {\sum\limits_{n =  - \infty }^\infty  {{X_a}\left( \Omega  \right)\delta \left( {\left( {\Omega T - \omega } \right) - n2\pi } \right)} d\Omega }.
\end{equation}

Simplify,

\begin{equation}
\begin{aligned}
X\left( \omega  \right) &= \frac{1}{T}\sum\limits_{n =  - \infty }^\infty  {\int\limits_{ - \infty }^\infty  {{X_a}\left( \Omega  \right)\delta \left( {\Omega  - \frac{\omega }{T} - \frac{{n2\pi }}{T}} \right)d\Omega } } \\
 &= \frac{1}{T}\sum\limits_{n =  - \infty }^\infty  {{X_a}\left( {\frac{\omega }{T} + \frac{{n2\pi }}{T}} \right)} .
\end{aligned}
\end{equation}
\end{proof}
