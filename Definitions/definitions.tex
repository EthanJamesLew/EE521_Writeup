\chapter{Definitions}


\section{Signals and Sequences}

\begin{defn}{\textbf{Kronecker Delta Function}}
    The Kronecker delta function, $\delta[n]$, is a sequence defined as

    \begin{equation}
        \delta [n] = \left\{ {\begin{array}{*{20}{c}}
            0&{n \ne 0}\\
            1&{n = 1}
            \end{array}} \right.,n \in \bN
    \end{equation}
\end{defn}

\textbf{Remark:}
\begin{equation}
    x[n] = \sum\limits_{k =  - \infty }^\infty  {x[k]\delta [n - k]} 
\end{equation}


\begin{defn}{\textbf{Periodic Sampling}}
A discrete-time representation $x[n]$ of a continuous signal $x_a(t)$ can be obtained from periodic sampling according to the relation
\begin{equation}
x[n] = x_a(nT)
\end{equation}
where $T$ is the sampling period and $f_s=1/T$ is the sampling frequency.
\end{defn}

\begin{defn}{\textbf{Linear Constant Coefficient Difference Equation}}
    A linear constant coefficient difference equation (LCCDE) is an equation of the form 
    \begin{equation}
        \sum\limits_{k=0}^{N} a_k y[n-k] = \sum\limits_{m=0}^{M} b_m x[n-m].
    \end{equation}

    Notably, linear time-invariant systems can have difference equation representations. 
\end{defn}

\section{Transforms}

\begin{defn}{\textbf{Continuous-Time Fourier Transform}}
If $x_a(t)$ is a function where $t \in \bR$, then its continuous-time Fourier transform $X_a(\Omega)$ is 
\begin{equation}
X_a(\Omega) = \int\limits_{ - \infty }^\infty  {{x_a}(t){e^{ - j\Omega t}}dt}.
\end{equation}

and its inverse transformation is

\begin{equation}
{x_a}\left( t \right) = \frac{1}{{2\pi }}\int\limits_{ - \infty }^\infty  {{X_a}(\Omega ){e^{ j\Omega t}}d\Omega }. 
\end{equation}
\end{defn}

\begin{defn}{\textbf{Discrete-Time Fourier Transform}}
If $x[n]$ is a function where $n \in \bN$, then its discrete-time Fourier transform $X(\omega)$ is 
\begin{equation}
X(\omega) = \sum\limits_{n =  - \infty }^\infty  {x[n]{e^{ - j\omega n}}}. 
\end{equation}
\end{defn}

DTFT General Properties

\begin{enumerate}
    \item $X(\omega)$ is a continuous function.
    \item $X(\omega)$ is a periodic function.
    \item Parseval's theorem holds that
     \begin{equation}
        \sum\limits_{n =  - \infty }^\infty  {{{\left| {x[n]} \right|}^2}}  = \frac{1}{{2\pi }}\int\limits_{ - \pi }^\pi  {{{\left| {X(\omega )} \right|}^2}d\omega } 
    \end{equation}
\end{enumerate}

\begin{defn}{\textbf{{Z-Transform}}}
If $x[n]$ is a function where $n \in \bN$, then its discrete-time Fourier transform $X(\omega)$ is 
\begin{equation}
    X(z) = \sum\limits_{n = -\infty}^\infty {x[n] z^{-n}}
\end{equation}
\end{defn}    

If $X(z)$ exists when $|z| = 1$, $X(e^{j \omega})$ is equal to the DTFT of $x[n]$. 

\textbf{Remark}: $Z\{ x[n - k] \} = z^{-k}(X(z) - \sum\limits_{i=-k}^{-1} x_i z^{-i}) $.

Z-Transform General Properties

\begin{enumerate}
\item 
\begin{equation}
    \begin{gathered}
        a{x_1}[n] + b{x_2}[n]\overset Z \longleftrightarrow a{X_1}(z) + b{X_2}(z) \\ 
        {\text{ROC}} = {R_{x1}}\bigcap {{R_{x2}}}  \\ 
      \end{gathered} 
\end{equation}

\item 
\begin{equation}
    \begin{gathered}
        x[n - {n_0}]\overset Z \longleftrightarrow {z^{ - {n_0}}}X(z) \\ 
        {\text{ROC}} = {R_x} \\ 
      \end{gathered} 
\end{equation}

\item 
\begin{equation}
\begin{gathered}
    nx[n]\overset Z \longleftrightarrow  - z\frac{{dx(z)}}{{dz}} \\ 
    {\text{ROC}} = {R_x} \\ 
  \end{gathered} 
\end{equation}

\item 
\begin{equation}
    \begin{gathered}
        {x^*}[n]\overset Z \longleftrightarrow {X^*}({z^*}) \\ 
        {\text{ROC}} = {R_x} \\ 
      \end{gathered} 
\end{equation}

\item 
\begin{equation}
    \begin{gathered}
        {x_1}[n] * {x_2}[n]\overset Z \longleftrightarrow {X_1}(z){X_2}(z) \\ 
        {\text{ROC}} = {R_{x1}}\bigcap {{R_{x2}}}  \\ 
      \end{gathered} 
\end{equation}

\begin{equation}
    \begin{gathered}
        y[n] = x[n] * h[n] = \sum\limits_{k = 0}^M {h[k]x[n - k]}  \\ 
         = \sum\limits_{k = 0}^M {h[k]} X(z){z^{ - k}} \\ 
         = X(z)H(z) \\ 
      \end{gathered} 
\end{equation}

\item 
\begin{equation}
    x[0] = \mathop {\lim }\limits_{z \to \infty } X(z),x[n] = 0{\text{ }}\forall n < 0 
\end{equation}

\end{enumerate}

\begin{defn}{\textbf{{Region of Convergence}}}
    The region of convergence (ROC), or more generally the radius of convergence, is a circle which 
    defines two sets of values, one of which converges under a transformation and one that diverges. 
\end{defn}  

ROC General Properties
\begin{enumerate}
    \item The ROC is a ring or disk centered at the origin of a z-plane.
    \item The ROC cannot contain any poles.
    \item If $x[n]$ is always finite than the ROC is the entire z-plane with the exception of $z=0$, or $z=\infty$.
    \item If $x[n]$ is right sided, the ROC extends from the outermost pole.
    \item If $x[n]$ is left sided, the ROC extends from the innermost pole to 0.
    \item If $x[n]$ is two-sided, the ROC is a ring bounded by two poles.
    \item The ROC is connected.
\end{enumerate}